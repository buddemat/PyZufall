% Generated by Sphinx.
\def\sphinxdocclass{report}
\documentclass[a4paper,12pt,oneside]{sphinxmanual}
\usepackage[utf8]{inputenc}
\DeclareUnicodeCharacter{00A0}{\nobreakspace}
\usepackage{cmap}
\usepackage[T1]{fontenc}
\usepackage[ngerman]{babel}
\usepackage{times}
\usepackage[Sonny]{fncychap}
\usepackage{longtable}
\usepackage{sphinx}
\usepackage{multirow}


\title{pyzufall Dokumentation}
\date{24. 07. 2013}
\release{0.8.0}
\author{davidak}
\newcommand{\sphinxlogo}{}
\renewcommand{\releasename}{Release}
\makeindex

\makeatletter
\def\PYG@reset{\let\PYG@it=\relax \let\PYG@bf=\relax%
    \let\PYG@ul=\relax \let\PYG@tc=\relax%
    \let\PYG@bc=\relax \let\PYG@ff=\relax}
\def\PYG@tok#1{\csname PYG@tok@#1\endcsname}
\def\PYG@toks#1+{\ifx\relax#1\empty\else%
    \PYG@tok{#1}\expandafter\PYG@toks\fi}
\def\PYG@do#1{\PYG@bc{\PYG@tc{\PYG@ul{%
    \PYG@it{\PYG@bf{\PYG@ff{#1}}}}}}}
\def\PYG#1#2{\PYG@reset\PYG@toks#1+\relax+\PYG@do{#2}}

\expandafter\def\csname PYG@tok@gd\endcsname{\def\PYG@tc##1{\textcolor[rgb]{0.63,0.00,0.00}{##1}}}
\expandafter\def\csname PYG@tok@gu\endcsname{\let\PYG@bf=\textbf\def\PYG@tc##1{\textcolor[rgb]{0.50,0.00,0.50}{##1}}}
\expandafter\def\csname PYG@tok@gt\endcsname{\def\PYG@tc##1{\textcolor[rgb]{0.00,0.27,0.87}{##1}}}
\expandafter\def\csname PYG@tok@gs\endcsname{\let\PYG@bf=\textbf}
\expandafter\def\csname PYG@tok@gr\endcsname{\def\PYG@tc##1{\textcolor[rgb]{1.00,0.00,0.00}{##1}}}
\expandafter\def\csname PYG@tok@cm\endcsname{\let\PYG@it=\textit\def\PYG@tc##1{\textcolor[rgb]{0.25,0.50,0.56}{##1}}}
\expandafter\def\csname PYG@tok@vg\endcsname{\def\PYG@tc##1{\textcolor[rgb]{0.73,0.38,0.84}{##1}}}
\expandafter\def\csname PYG@tok@m\endcsname{\def\PYG@tc##1{\textcolor[rgb]{0.13,0.50,0.31}{##1}}}
\expandafter\def\csname PYG@tok@mh\endcsname{\def\PYG@tc##1{\textcolor[rgb]{0.13,0.50,0.31}{##1}}}
\expandafter\def\csname PYG@tok@cs\endcsname{\def\PYG@tc##1{\textcolor[rgb]{0.25,0.50,0.56}{##1}}\def\PYG@bc##1{\setlength{\fboxsep}{0pt}\colorbox[rgb]{1.00,0.94,0.94}{\strut ##1}}}
\expandafter\def\csname PYG@tok@ge\endcsname{\let\PYG@it=\textit}
\expandafter\def\csname PYG@tok@vc\endcsname{\def\PYG@tc##1{\textcolor[rgb]{0.73,0.38,0.84}{##1}}}
\expandafter\def\csname PYG@tok@il\endcsname{\def\PYG@tc##1{\textcolor[rgb]{0.13,0.50,0.31}{##1}}}
\expandafter\def\csname PYG@tok@go\endcsname{\def\PYG@tc##1{\textcolor[rgb]{0.20,0.20,0.20}{##1}}}
\expandafter\def\csname PYG@tok@cp\endcsname{\def\PYG@tc##1{\textcolor[rgb]{0.00,0.44,0.13}{##1}}}
\expandafter\def\csname PYG@tok@gi\endcsname{\def\PYG@tc##1{\textcolor[rgb]{0.00,0.63,0.00}{##1}}}
\expandafter\def\csname PYG@tok@gh\endcsname{\let\PYG@bf=\textbf\def\PYG@tc##1{\textcolor[rgb]{0.00,0.00,0.50}{##1}}}
\expandafter\def\csname PYG@tok@ni\endcsname{\let\PYG@bf=\textbf\def\PYG@tc##1{\textcolor[rgb]{0.84,0.33,0.22}{##1}}}
\expandafter\def\csname PYG@tok@nl\endcsname{\let\PYG@bf=\textbf\def\PYG@tc##1{\textcolor[rgb]{0.00,0.13,0.44}{##1}}}
\expandafter\def\csname PYG@tok@nn\endcsname{\let\PYG@bf=\textbf\def\PYG@tc##1{\textcolor[rgb]{0.05,0.52,0.71}{##1}}}
\expandafter\def\csname PYG@tok@no\endcsname{\def\PYG@tc##1{\textcolor[rgb]{0.38,0.68,0.84}{##1}}}
\expandafter\def\csname PYG@tok@na\endcsname{\def\PYG@tc##1{\textcolor[rgb]{0.25,0.44,0.63}{##1}}}
\expandafter\def\csname PYG@tok@nb\endcsname{\def\PYG@tc##1{\textcolor[rgb]{0.00,0.44,0.13}{##1}}}
\expandafter\def\csname PYG@tok@nc\endcsname{\let\PYG@bf=\textbf\def\PYG@tc##1{\textcolor[rgb]{0.05,0.52,0.71}{##1}}}
\expandafter\def\csname PYG@tok@nd\endcsname{\let\PYG@bf=\textbf\def\PYG@tc##1{\textcolor[rgb]{0.33,0.33,0.33}{##1}}}
\expandafter\def\csname PYG@tok@ne\endcsname{\def\PYG@tc##1{\textcolor[rgb]{0.00,0.44,0.13}{##1}}}
\expandafter\def\csname PYG@tok@nf\endcsname{\def\PYG@tc##1{\textcolor[rgb]{0.02,0.16,0.49}{##1}}}
\expandafter\def\csname PYG@tok@si\endcsname{\let\PYG@it=\textit\def\PYG@tc##1{\textcolor[rgb]{0.44,0.63,0.82}{##1}}}
\expandafter\def\csname PYG@tok@s2\endcsname{\def\PYG@tc##1{\textcolor[rgb]{0.25,0.44,0.63}{##1}}}
\expandafter\def\csname PYG@tok@vi\endcsname{\def\PYG@tc##1{\textcolor[rgb]{0.73,0.38,0.84}{##1}}}
\expandafter\def\csname PYG@tok@nt\endcsname{\let\PYG@bf=\textbf\def\PYG@tc##1{\textcolor[rgb]{0.02,0.16,0.45}{##1}}}
\expandafter\def\csname PYG@tok@nv\endcsname{\def\PYG@tc##1{\textcolor[rgb]{0.73,0.38,0.84}{##1}}}
\expandafter\def\csname PYG@tok@s1\endcsname{\def\PYG@tc##1{\textcolor[rgb]{0.25,0.44,0.63}{##1}}}
\expandafter\def\csname PYG@tok@gp\endcsname{\let\PYG@bf=\textbf\def\PYG@tc##1{\textcolor[rgb]{0.78,0.36,0.04}{##1}}}
\expandafter\def\csname PYG@tok@sh\endcsname{\def\PYG@tc##1{\textcolor[rgb]{0.25,0.44,0.63}{##1}}}
\expandafter\def\csname PYG@tok@ow\endcsname{\let\PYG@bf=\textbf\def\PYG@tc##1{\textcolor[rgb]{0.00,0.44,0.13}{##1}}}
\expandafter\def\csname PYG@tok@sx\endcsname{\def\PYG@tc##1{\textcolor[rgb]{0.78,0.36,0.04}{##1}}}
\expandafter\def\csname PYG@tok@bp\endcsname{\def\PYG@tc##1{\textcolor[rgb]{0.00,0.44,0.13}{##1}}}
\expandafter\def\csname PYG@tok@c1\endcsname{\let\PYG@it=\textit\def\PYG@tc##1{\textcolor[rgb]{0.25,0.50,0.56}{##1}}}
\expandafter\def\csname PYG@tok@kc\endcsname{\let\PYG@bf=\textbf\def\PYG@tc##1{\textcolor[rgb]{0.00,0.44,0.13}{##1}}}
\expandafter\def\csname PYG@tok@c\endcsname{\let\PYG@it=\textit\def\PYG@tc##1{\textcolor[rgb]{0.25,0.50,0.56}{##1}}}
\expandafter\def\csname PYG@tok@mf\endcsname{\def\PYG@tc##1{\textcolor[rgb]{0.13,0.50,0.31}{##1}}}
\expandafter\def\csname PYG@tok@err\endcsname{\def\PYG@bc##1{\setlength{\fboxsep}{0pt}\fcolorbox[rgb]{1.00,0.00,0.00}{1,1,1}{\strut ##1}}}
\expandafter\def\csname PYG@tok@kd\endcsname{\let\PYG@bf=\textbf\def\PYG@tc##1{\textcolor[rgb]{0.00,0.44,0.13}{##1}}}
\expandafter\def\csname PYG@tok@ss\endcsname{\def\PYG@tc##1{\textcolor[rgb]{0.32,0.47,0.09}{##1}}}
\expandafter\def\csname PYG@tok@sr\endcsname{\def\PYG@tc##1{\textcolor[rgb]{0.14,0.33,0.53}{##1}}}
\expandafter\def\csname PYG@tok@mo\endcsname{\def\PYG@tc##1{\textcolor[rgb]{0.13,0.50,0.31}{##1}}}
\expandafter\def\csname PYG@tok@mi\endcsname{\def\PYG@tc##1{\textcolor[rgb]{0.13,0.50,0.31}{##1}}}
\expandafter\def\csname PYG@tok@kn\endcsname{\let\PYG@bf=\textbf\def\PYG@tc##1{\textcolor[rgb]{0.00,0.44,0.13}{##1}}}
\expandafter\def\csname PYG@tok@o\endcsname{\def\PYG@tc##1{\textcolor[rgb]{0.40,0.40,0.40}{##1}}}
\expandafter\def\csname PYG@tok@kr\endcsname{\let\PYG@bf=\textbf\def\PYG@tc##1{\textcolor[rgb]{0.00,0.44,0.13}{##1}}}
\expandafter\def\csname PYG@tok@s\endcsname{\def\PYG@tc##1{\textcolor[rgb]{0.25,0.44,0.63}{##1}}}
\expandafter\def\csname PYG@tok@kp\endcsname{\def\PYG@tc##1{\textcolor[rgb]{0.00,0.44,0.13}{##1}}}
\expandafter\def\csname PYG@tok@w\endcsname{\def\PYG@tc##1{\textcolor[rgb]{0.73,0.73,0.73}{##1}}}
\expandafter\def\csname PYG@tok@kt\endcsname{\def\PYG@tc##1{\textcolor[rgb]{0.56,0.13,0.00}{##1}}}
\expandafter\def\csname PYG@tok@sc\endcsname{\def\PYG@tc##1{\textcolor[rgb]{0.25,0.44,0.63}{##1}}}
\expandafter\def\csname PYG@tok@sb\endcsname{\def\PYG@tc##1{\textcolor[rgb]{0.25,0.44,0.63}{##1}}}
\expandafter\def\csname PYG@tok@k\endcsname{\let\PYG@bf=\textbf\def\PYG@tc##1{\textcolor[rgb]{0.00,0.44,0.13}{##1}}}
\expandafter\def\csname PYG@tok@se\endcsname{\let\PYG@bf=\textbf\def\PYG@tc##1{\textcolor[rgb]{0.25,0.44,0.63}{##1}}}
\expandafter\def\csname PYG@tok@sd\endcsname{\let\PYG@it=\textit\def\PYG@tc##1{\textcolor[rgb]{0.25,0.44,0.63}{##1}}}

\def\PYGZbs{\char`\\}
\def\PYGZus{\char`\_}
\def\PYGZob{\char`\{}
\def\PYGZcb{\char`\}}
\def\PYGZca{\char`\^}
\def\PYGZam{\char`\&}
\def\PYGZlt{\char`\<}
\def\PYGZgt{\char`\>}
\def\PYGZsh{\char`\#}
\def\PYGZpc{\char`\%}
\def\PYGZdl{\char`\$}
\def\PYGZhy{\char`\-}
\def\PYGZsq{\char`\'}
\def\PYGZdq{\char`\"}
\def\PYGZti{\char`\~}
% for compatibility with earlier versions
\def\PYGZat{@}
\def\PYGZlb{[}
\def\PYGZrb{]}
\makeatother

\begin{document}
\shorthandoff{"}
\maketitle
\tableofcontents
\phantomsection\label{index::doc}



\chapter{Aufgaben}
\label{todo::doc}\label{todo:dokumentation-von-pyzufall}\label{todo:aufgaben}
Für Fehlerberichte und Feature-Requests wird der \href{https://github.com/davidak/pyzufall/issues}{Bugtracker auf github} benutzt.

Auch im Quelltext gibt es Hinweise auf nötige Anpassungen:

\begin{notice}{note}{Zu tun}

Sollte in zwei Funktionen aufgeteilt werden.
\end{notice}

(Der {\hyperref[funktionen:index-0]{\emph{ursprüngliche Eintrag}}} steht in /Users/davidak/BTSync/code/pyzufall/pyzufall.py:docstring of pyzufall.essen, Zeile 3.)

\begin{notice}{note}{Zu tun}

Dokuwiki auf satzgenerator.de/beitragen einrichten mit Kopie der Datensätze. Bearbeiten nach Registrierung möglich.
\end{notice}

(Der {\hyperref[index:index-0]{\emph{ursprüngliche Eintrag}}} steht in /Users/davidak/BTSync/code/pyzufall/doc/index.rst, Zeile 81.)


\chapter{Funktions Referenz}
\label{funktionen:funktions-referenz}\label{funktionen::doc}\label{funktionen:module-pyzufall}\index{pyzufall (Modul)}
Generiert unter anderem Namen, Orte, Fantasieworte, Berufsbezeichnungen und letztendlich ganze Sätze.
\index{adj() (in Modul pyzufall)}

\begin{fulllineitems}
\phantomsection\label{funktionen:pyzufall.adj}\pysiglinewithargsret{\code{pyzufall.}\bfcode{adj}}{}{}
Gibt ein Adjektiv zurück.

\end{fulllineitems}

\index{band() (in Modul pyzufall)}

\begin{fulllineitems}
\phantomsection\label{funktionen:pyzufall.band}\pysiglinewithargsret{\code{pyzufall.}\bfcode{band}}{}{}
Gibt einen fiktiven Bandnamen zurück.

\end{fulllineitems}

\index{bandart() (in Modul pyzufall)}

\begin{fulllineitems}
\phantomsection\label{funktionen:pyzufall.bandart}\pysiglinewithargsret{\code{pyzufall.}\bfcode{bandart}}{}{}
Gibt eine Bandart zurück.

Beispiel: `Gothic Metal Band'

\end{fulllineitems}

\index{baum() (in Modul pyzufall)}

\begin{fulllineitems}
\phantomsection\label{funktionen:pyzufall.baum}\pysiglinewithargsret{\code{pyzufall.}\bfcode{baum}}{}{}
Gibt einen Baum zurück.

\end{fulllineitems}

\index{beilage() (in Modul pyzufall)}

\begin{fulllineitems}
\phantomsection\label{funktionen:pyzufall.beilage}\pysiglinewithargsret{\code{pyzufall.}\bfcode{beilage}}{}{}
Gibt eine Beilage zum Essen zurück.

\end{fulllineitems}

\index{beruf\_m() (in Modul pyzufall)}

\begin{fulllineitems}
\phantomsection\label{funktionen:pyzufall.beruf_m}\pysiglinewithargsret{\code{pyzufall.}\bfcode{beruf\_m}}{}{}
Gibt eine männliche Berufsbezeichnung zurück.

\end{fulllineitems}

\index{beruf\_w() (in Modul pyzufall)}

\begin{fulllineitems}
\phantomsection\label{funktionen:pyzufall.beruf_w}\pysiglinewithargsret{\code{pyzufall.}\bfcode{beruf\_w}}{}{}
Gibt eine weibliche Berufsbezeichnung zurück.

\end{fulllineitems}

\index{color() (in Modul pyzufall)}

\begin{fulllineitems}
\phantomsection\label{funktionen:pyzufall.color}\pysiglinewithargsret{\code{pyzufall.}\bfcode{color}}{}{}
Gibt eine Farbe auf englisch zurück.

\end{fulllineitems}

\index{datum() (in Modul pyzufall)}

\begin{fulllineitems}
\phantomsection\label{funktionen:pyzufall.datum}\pysiglinewithargsret{\code{pyzufall.}\bfcode{datum}}{}{}
Gibt ein Datum zwischen dem 01.01.1950 und 31.12.2012 zurück.

\end{fulllineitems}

\index{e16() (in Modul pyzufall)}

\begin{fulllineitems}
\phantomsection\label{funktionen:pyzufall.e16}\pysiglinewithargsret{\code{pyzufall.}\bfcode{e16}}{\emph{wort}}{}
Das übergebene Wort wird mit einer Wahrscheinlichkeit von 16\% zurückgegeben.

\end{fulllineitems}

\index{e25() (in Modul pyzufall)}

\begin{fulllineitems}
\phantomsection\label{funktionen:pyzufall.e25}\pysiglinewithargsret{\code{pyzufall.}\bfcode{e25}}{\emph{wort}}{}
Das übergebene Wort wird mit einer Wahrscheinlichkeit von 25\% zurückgegeben.

\end{fulllineitems}

\index{e50() (in Modul pyzufall)}

\begin{fulllineitems}
\phantomsection\label{funktionen:pyzufall.e50}\pysiglinewithargsret{\code{pyzufall.}\bfcode{e50}}{\emph{wort}}{}
Das übergebene Wort wird mit einer Wahrscheinlichkeit von 50\% zurückgegeben.

\end{fulllineitems}

\index{e75() (in Modul pyzufall)}

\begin{fulllineitems}
\phantomsection\label{funktionen:pyzufall.e75}\pysiglinewithargsret{\code{pyzufall.}\bfcode{e75}}{\emph{wort}}{}
Das übergebene Wort wird mit einer Wahrscheinlichkeit von 75\% zurückgegeben.

\end{fulllineitems}

\index{ersten\_buchstaben\_gross() (in Modul pyzufall)}

\begin{fulllineitems}
\phantomsection\label{funktionen:pyzufall.ersten_buchstaben_gross}\pysiglinewithargsret{\code{pyzufall.}\bfcode{ersten\_buchstaben\_gross}}{\emph{s}}{}
Macht den ersten Buchstaben gross.

\end{fulllineitems}

\index{essen() (in Modul pyzufall)}

\begin{fulllineitems}
\phantomsection\label{funktionen:pyzufall.essen}\pysiglinewithargsret{\code{pyzufall.}\bfcode{essen}}{\emph{anz}}{}
Gibt Essen zurück.

\begin{notice}{note}{Zu tun}

Sollte in zwei Funktionen aufgeteilt werden.
\end{notice}

\end{fulllineitems}

\index{farbe() (in Modul pyzufall)}

\begin{fulllineitems}
\phantomsection\label{funktionen:pyzufall.farbe}\pysiglinewithargsret{\code{pyzufall.}\bfcode{farbe}}{}{}
Gibt eine Farbe zurück.

\end{fulllineitems}

\index{firma() (in Modul pyzufall)}

\begin{fulllineitems}
\phantomsection\label{funktionen:pyzufall.firma}\pysiglinewithargsret{\code{pyzufall.}\bfcode{firma}}{}{}
Gibt einen fiktiven Firmenname zurück.

TODO

\end{fulllineitems}

\index{frage() (in Modul pyzufall)}

\begin{fulllineitems}
\phantomsection\label{funktionen:pyzufall.frage}\pysiglinewithargsret{\code{pyzufall.}\bfcode{frage}}{}{}
Gibt eine Frage zurück.

\end{fulllineitems}

\index{gegenstand() (in Modul pyzufall)}

\begin{fulllineitems}
\phantomsection\label{funktionen:pyzufall.gegenstand}\pysiglinewithargsret{\code{pyzufall.}\bfcode{gegenstand}}{}{}
Gibt einen Gegenstand zurück.

\end{fulllineitems}

\index{koerperteil() (in Modul pyzufall)}

\begin{fulllineitems}
\phantomsection\label{funktionen:pyzufall.koerperteil}\pysiglinewithargsret{\code{pyzufall.}\bfcode{koerperteil}}{}{}
Gibt ein Körperteil zurück.

\end{fulllineitems}

\index{lese() (in Modul pyzufall)}

\begin{fulllineitems}
\phantomsection\label{funktionen:pyzufall.lese}\pysiglinewithargsret{\code{pyzufall.}\bfcode{lese}}{\emph{dateiname}}{}
Liest die Datei mit dem übergebenen Namen aus data/ zeilenweise ein und gib eine Liste zurück.

\href{http://stackoverflow.com/questions/10174211/make-an-always-relative-to-current-module-file-path}{http://stackoverflow.com/questions/10174211/make-an-always-relative-to-current-module-file-path}
\href{http://stackoverflow.com/questions/595305/python-path-of-scrip}{http://stackoverflow.com/questions/595305/python-path-of-scrip}

\end{fulllineitems}

\index{nachname() (in Modul pyzufall)}

\begin{fulllineitems}
\phantomsection\label{funktionen:pyzufall.nachname}\pysiglinewithargsret{\code{pyzufall.}\bfcode{nachname}}{}{}
Gibt einen Nachnamen zurück.

\end{fulllineitems}

\index{objekt() (in Modul pyzufall)}

\begin{fulllineitems}
\phantomsection\label{funktionen:pyzufall.objekt}\pysiglinewithargsret{\code{pyzufall.}\bfcode{objekt}}{}{}
Gibt ein Objekt zurück.

\end{fulllineitems}

\index{objekt\_m() (in Modul pyzufall)}

\begin{fulllineitems}
\phantomsection\label{funktionen:pyzufall.objekt_m}\pysiglinewithargsret{\code{pyzufall.}\bfcode{objekt\_m}}{\emph{s}}{}
Bringt ein Objekt in Berzug zu einer männlichen Person.

Beispiel:
`der Bär' wird zu `den Bären' oder `seinen Bären'

\end{fulllineitems}

\index{objekt\_w() (in Modul pyzufall)}

\begin{fulllineitems}
\phantomsection\label{funktionen:pyzufall.objekt_w}\pysiglinewithargsret{\code{pyzufall.}\bfcode{objekt\_w}}{\emph{s}}{}
Bringt ein Objekt in Berzug zu einer weiblichen Person.

Beispiel:
`der Bär' wird zu `den Bären' oder `ihren Bären'

\end{fulllineitems}

\index{ort() (in Modul pyzufall)}

\begin{fulllineitems}
\phantomsection\label{funktionen:pyzufall.ort}\pysiglinewithargsret{\code{pyzufall.}\bfcode{ort}}{}{}
Gibt eine Ortsangabe zurück.

Beispiel: `im Flur'

\end{fulllineitems}

\index{person() (in Modul pyzufall)}

\begin{fulllineitems}
\phantomsection\label{funktionen:pyzufall.person}\pysiglinewithargsret{\code{pyzufall.}\bfcode{person}}{}{}
Gibt eine Person zurück.

\end{fulllineitems}

\index{person\_m() (in Modul pyzufall)}

\begin{fulllineitems}
\phantomsection\label{funktionen:pyzufall.person_m}\pysiglinewithargsret{\code{pyzufall.}\bfcode{person\_m}}{}{}
Gibt eine männliche Person zurück.

\end{fulllineitems}

\index{person\_objekt\_m() (in Modul pyzufall)}

\begin{fulllineitems}
\phantomsection\label{funktionen:pyzufall.person_objekt_m}\pysiglinewithargsret{\code{pyzufall.}\bfcode{person\_objekt\_m}}{}{}
Gibt eine Person als Objekt in Bezug auf eine männliche Person zurück.

\end{fulllineitems}

\index{person\_objekt\_w() (in Modul pyzufall)}

\begin{fulllineitems}
\phantomsection\label{funktionen:pyzufall.person_objekt_w}\pysiglinewithargsret{\code{pyzufall.}\bfcode{person\_objekt\_w}}{}{}
Gibt eine Person als Objekt in Bezug auf eine weibliche Person zurück.

\end{fulllineitems}

\index{person\_w() (in Modul pyzufall)}

\begin{fulllineitems}
\phantomsection\label{funktionen:pyzufall.person_w}\pysiglinewithargsret{\code{pyzufall.}\bfcode{person\_w}}{}{}
Gibt eine weibliche Person zurück.

\end{fulllineitems}

\index{pflanze() (in Modul pyzufall)}

\begin{fulllineitems}
\phantomsection\label{funktionen:pyzufall.pflanze}\pysiglinewithargsret{\code{pyzufall.}\bfcode{pflanze}}{}{}
Gibt eine Pflanze zurück.

\end{fulllineitems}

\index{satz() (in Modul pyzufall)}

\begin{fulllineitems}
\phantomsection\label{funktionen:pyzufall.satz}\pysiglinewithargsret{\code{pyzufall.}\bfcode{satz}}{}{}
Gibt einen Satz zurück.

\end{fulllineitems}

\index{sprichwort() (in Modul pyzufall)}

\begin{fulllineitems}
\phantomsection\label{funktionen:pyzufall.sprichwort}\pysiglinewithargsret{\code{pyzufall.}\bfcode{sprichwort}}{}{}
Gibt ein Sprichwort zurück.

\end{fulllineitems}

\index{stadt() (in Modul pyzufall)}

\begin{fulllineitems}
\phantomsection\label{funktionen:pyzufall.stadt}\pysiglinewithargsret{\code{pyzufall.}\bfcode{stadt}}{}{}
Gibt eine Stadt zurück.

\end{fulllineitems}

\index{stadt\_bl() (in Modul pyzufall)}

\begin{fulllineitems}
\phantomsection\label{funktionen:pyzufall.stadt_bl}\pysiglinewithargsret{\code{pyzufall.}\bfcode{stadt\_bl}}{}{}
Gibt eine Stadt mit Bundesland zurück.

\end{fulllineitems}

\index{standard\_satz() (in Modul pyzufall)}

\begin{fulllineitems}
\phantomsection\label{funktionen:pyzufall.standard_satz}\pysiglinewithargsret{\code{pyzufall.}\bfcode{standard\_satz}}{}{}
Gibt einen einfachen Satz, bestehend aus Person, Verb, Adjektiv, Ort und teilweise auch Objekt zurück.

\end{fulllineitems}

\index{themen\_satz() (in Modul pyzufall)}

\begin{fulllineitems}
\phantomsection\label{funktionen:pyzufall.themen_satz}\pysiglinewithargsret{\code{pyzufall.}\bfcode{themen\_satz}}{}{}
Gibt einen Satz zu einem zufälligen Themen-Komplex zurück.

\end{fulllineitems}

\index{tier() (in Modul pyzufall)}

\begin{fulllineitems}
\phantomsection\label{funktionen:pyzufall.tier}\pysiglinewithargsret{\code{pyzufall.}\bfcode{tier}}{}{}
Gibt ein Tier zurück.

\end{fulllineitems}

\index{trinken() (in Modul pyzufall)}

\begin{fulllineitems}
\phantomsection\label{funktionen:pyzufall.trinken}\pysiglinewithargsret{\code{pyzufall.}\bfcode{trinken}}{}{}
Gibt ein Getränk zurück.

\end{fulllineitems}

\index{verbd() (in Modul pyzufall)}

\begin{fulllineitems}
\phantomsection\label{funktionen:pyzufall.verbd}\pysiglinewithargsret{\code{pyzufall.}\bfcode{verbd}}{}{}
Gibt ein ditransitives Verb zurück.

\href{http://de.wikipedia.org/wiki}{http://de.wikipedia.org/wiki}/Transitivität\_(Grammatik)\#Festlegung\_der\_Transitivit.C3.A4t\_eines\_Verbs

\end{fulllineitems}

\index{verbi() (in Modul pyzufall)}

\begin{fulllineitems}
\phantomsection\label{funktionen:pyzufall.verbi}\pysiglinewithargsret{\code{pyzufall.}\bfcode{verbi}}{}{}
Gibt ein intransitives Verb zurück.

\href{http://de.wikipedia.org/wiki}{http://de.wikipedia.org/wiki}/Transitivität\_(Grammatik)\#Festlegung\_der\_Transitivit.C3.A4t\_eines\_Verbs

\end{fulllineitems}

\index{verbi2() (in Modul pyzufall)}

\begin{fulllineitems}
\phantomsection\label{funktionen:pyzufall.verbi2}\pysiglinewithargsret{\code{pyzufall.}\bfcode{verbi2}}{}{}
Gibt ein intransitives, getrenntes Verb zurück.

\href{http://de.wikipedia.org/wiki}{http://de.wikipedia.org/wiki}/Transitivität\_(Grammatik)\#Festlegung\_der\_Transitivit.C3.A4t\_eines\_Verbs

\end{fulllineitems}

\index{verbn() (in Modul pyzufall)}

\begin{fulllineitems}
\phantomsection\label{funktionen:pyzufall.verbn}\pysiglinewithargsret{\code{pyzufall.}\bfcode{verbn}}{}{}
Gibt ein nullwertiges Verb zurück.

\href{http://de.wikipedia.org/wiki}{http://de.wikipedia.org/wiki}/Transitivität\_(Grammatik)\#Festlegung\_der\_Transitivit.C3.A4t\_eines\_Verbs

\end{fulllineitems}

\index{verbt() (in Modul pyzufall)}

\begin{fulllineitems}
\phantomsection\label{funktionen:pyzufall.verbt}\pysiglinewithargsret{\code{pyzufall.}\bfcode{verbt}}{}{}
Gibt ein transitives Verb zurück.

\href{http://de.wikipedia.org/wiki}{http://de.wikipedia.org/wiki}/Transitivität\_(Grammatik)\#Festlegung\_der\_Transitivit.C3.A4t\_eines\_Verbs

\end{fulllineitems}

\index{verbt2() (in Modul pyzufall)}

\begin{fulllineitems}
\phantomsection\label{funktionen:pyzufall.verbt2}\pysiglinewithargsret{\code{pyzufall.}\bfcode{verbt2}}{}{}
Gibt ein intransitives, getrenntes Verb zurück.

\href{http://de.wikipedia.org/wiki}{http://de.wikipedia.org/wiki}/Transitivität\_(Grammatik)\#Festlegung\_der\_Transitivit.C3.A4t\_eines\_Verbs

\end{fulllineitems}

\index{vorname\_m() (in Modul pyzufall)}

\begin{fulllineitems}
\phantomsection\label{funktionen:pyzufall.vorname_m}\pysiglinewithargsret{\code{pyzufall.}\bfcode{vorname\_m}}{}{}
Gibt einen männlichen Vornamen zurück.

\end{fulllineitems}

\index{vorname\_w() (in Modul pyzufall)}

\begin{fulllineitems}
\phantomsection\label{funktionen:pyzufall.vorname_w}\pysiglinewithargsret{\code{pyzufall.}\bfcode{vorname\_w}}{}{}
Gibt einen weiblichen Vornamen zurück.

\end{fulllineitems}

\index{wort() (in Modul pyzufall)}

\begin{fulllineitems}
\phantomsection\label{funktionen:pyzufall.wort}\pysiglinewithargsret{\code{pyzufall.}\bfcode{wort}}{}{}
Gibt ein Fantasiewort zurück.

\end{fulllineitems}

\index{zahl() (in Modul pyzufall)}

\begin{fulllineitems}
\phantomsection\label{funktionen:pyzufall.zahl}\pysiglinewithargsret{\code{pyzufall.}\bfcode{zahl}}{}{}
Gibt eine Zahl zwischen 0 und 100 zurück.

\end{fulllineitems}


Das Python-Modul {\hyperref[funktionen:module-pyzufall]{\code{pyzufall}}} hat viele Funktionen, um diverse Daten zu erzeugen.

Die komplexeste ist dabei {\hyperref[funktionen:pyzufall.satz]{\code{pyzufall.satz()}}}, mit der sich zufällige Sätze generieren lassen.

Es wurde für \href{http://satzgenerator.de/}{satzgenerator.de} entwickelt, kann aber vielfältig eingesetzt werden.


\chapter{Installation}
\label{index:installation}
Die aktuellste Version findest du auf \href{https://github.com/davidak/pyzufall}{github}. Eine stabile Version wird es auf \href{https://pypi.python.org/}{PyPI} geben.

Klone das git-Repository in dein Projekt-Verzeichnis:

\begin{Verbatim}[commandchars=\\\{\}]
\$ mkdir meinprojekt
\$ cd meinprojekt
\$ git clone https://github.com/davidak/pyzufall.git
\end{Verbatim}


\chapter{Verwendung}
\label{index:verwendung}
Um {\hyperref[funktionen:module-pyzufall]{\code{pyzufall}}} verwenden zu können, muss es importiert werden:

\begin{Verbatim}[commandchars=\\\{\}]
\PYG{g+gp}{\PYGZgt{}\PYGZgt{}\PYGZgt{} }\PYG{k+kn}{import} \PYG{n+nn}{pyzufall} \PYG{k+kn}{as} \PYG{n+nn}{z}
\end{Verbatim}


\chapter{Beispiele}
\label{index:beispiele}
Einen zufälligen männlichen Vornamen erzeugen:

\begin{Verbatim}[commandchars=\\\{\}]
\PYG{g+gp}{\PYGZgt{}\PYGZgt{}\PYGZgt{} }\PYG{k}{print}\PYG{p}{(}\PYG{n}{z}\PYG{o}{.}\PYG{n}{vorname\PYGZus{}m}\PYG{p}{(}\PYG{p}{)}\PYG{p}{)}
\PYG{g+go}{Magnus}
\end{Verbatim}

Einen zufälligen weiblichen Vorname mit Nachname erzeugen:

\begin{Verbatim}[commandchars=\\\{\}]
\PYG{g+gp}{\PYGZgt{}\PYGZgt{}\PYGZgt{} }\PYG{k}{print}\PYG{p}{(}\PYG{n}{z}\PYG{o}{.}\PYG{n}{vorname\PYGZus{}w}\PYG{p}{(}\PYG{p}{)} \PYG{o}{+} \PYG{l+s}{\PYGZsq{}}\PYG{l+s}{ }\PYG{l+s}{\PYGZsq{}} \PYG{o}{+} \PYG{n}{z}\PYG{o}{.}\PYG{n}{nachname}\PYG{p}{(}\PYG{p}{)}\PYG{p}{)}
\PYG{g+go}{Carmen Büchler}
\end{Verbatim}

Ein zufälliges Sprichwort erzeugen:

\begin{Verbatim}[commandchars=\\\{\}]
\PYG{g+gp}{\PYGZgt{}\PYGZgt{}\PYGZgt{} }\PYG{k}{print}\PYG{p}{(}\PYG{n}{z}\PYG{o}{.}\PYG{n}{sprichwort}\PYG{p}{(}\PYG{p}{)}\PYG{p}{)}
\PYG{g+go}{Das schlägt dem Fass den Boden aus.}
\end{Verbatim}

Die Funktion {\hyperref[funktionen:pyzufall.satz]{\code{pyzufall.satz()}}} benutzt die meisten anderen Funktionen, um ganze Sätze zu generieren.
Es sind viele Satz-Schemata hinterlegt für abwechlungsreiche Ergebnisse.

Hier einige Beispiele:

\begin{Verbatim}[commandchars=\\\{\}]
\PYG{g+gp}{\PYGZgt{}\PYGZgt{}\PYGZgt{} }\PYG{k}{print}\PYG{p}{(}\PYG{n}{z}\PYG{o}{.}\PYG{n}{satz}\PYG{p}{(}\PYG{p}{)}\PYG{p}{)}
\PYG{g+go}{Weshalb katasysiert der witzige Wolfram fantasielos unter der Brücke?}
\PYG{g+gp}{\PYGZgt{}\PYGZgt{}\PYGZgt{} }\PYG{k}{print}\PYG{p}{(}\PYG{n}{z}\PYG{o}{.}\PYG{n}{satz}\PYG{p}{(}\PYG{p}{)}\PYG{p}{)}
\PYG{g+go}{Die Lehrerin zersägt deine Rosskastanie.}
\PYG{g+gp}{\PYGZgt{}\PYGZgt{}\PYGZgt{} }\PYG{k}{print}\PYG{p}{(}\PYG{n}{z}\PYG{o}{.}\PYG{n}{satz}\PYG{p}{(}\PYG{p}{)}\PYG{p}{)}
\PYG{g+go}{Der Kollege programmiert deine Partnerin im Atomkraftwerk.}
\PYG{g+gp}{\PYGZgt{}\PYGZgt{}\PYGZgt{} }\PYG{k}{print}\PYG{p}{(}\PYG{n}{z}\PYG{o}{.}\PYG{n}{satz}\PYG{p}{(}\PYG{p}{)}\PYG{p}{)}
\PYG{g+go}{Heinrich gewinnt den Ahorn heimtückisch auf einer Hochzeit.}
\end{Verbatim}

Eine Übersicht aller Funktionen findest du in der Referenz:  {\hyperref[funktionen:module-pyzufall]{\code{pyzufall}}}


\chapter{Beitragen}
\label{index:beitragen}
Du kannst bei diesem Open Source-Projekt mitwirken, indem du \href{https://github.com/davidak/pyzufall/issues/}{Fehler berichtest}, neue Datensätze hinzufügst oder sogar mitprogrammierst.

Die Vielfalt und Anzahl der möglichen Sätze steigt mit den Datensätzen. An einer einfachen Möglichkeit, Daten hinzuzufügen, wird gearbeitet.

\begin{notice}{note}{Zu tun}

Dokuwiki auf satzgenerator.de/beitragen einrichten mit Kopie der Datensätze. Bearbeiten nach Registrierung möglich.
\end{notice}

Pull-Requests auf github sind willkommen.


\chapter{Benutzer}
\label{index:benutzer}
Hier ist eine Liste mit Projekten, die {\hyperref[funktionen:module-pyzufall]{\code{pyzufall}}} verwenden:
\begin{itemize}
\item {} 
\href{http://satzgenerator.de/}{satzgenerator.de}

\item {} 
\href{https://github.com/davidak/python-random-vcard-generator}{Python Random VCard-Generator}

\end{itemize}

Dein Projekt füge ich auch gerne hinzu.

Auch über Rückmeldungen jeder Art freue ich mich.

Einfach eine E-Mail an post at davidak punkt de oder das \href{http://davidak.de/kontakt}{Kontaktformular} benutzen.


\renewcommand{\indexname}{Python-Modulindex}
\begin{theindex}
\def\bigletter#1{{\Large\sffamily#1}\nopagebreak\vspace{1mm}}
\bigletter{p}
\item {\texttt{pyzufall}}, \pageref{funktionen:module-pyzufall}
\end{theindex}

\renewcommand{\indexname}{Stichwortverzeichnis}
\printindex
\end{document}

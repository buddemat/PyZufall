% Generated by Sphinx.
\def\sphinxdocclass{report}
\documentclass[a4paper,12pt,oneside]{sphinxmanual}
\usepackage[utf8]{inputenc}
\DeclareUnicodeCharacter{00A0}{\nobreakspace}
\usepackage{cmap}
\usepackage[T1]{fontenc}
\usepackage[ngerman]{babel}
\usepackage{times}
\usepackage[Sonny]{fncychap}
\usepackage{longtable}
\usepackage{sphinx}
\usepackage{multirow}


\title{pyzufall Dokumentation}
\date{13. 09. 2013}
\release{0.10.1}
\author{davidak}
\newcommand{\sphinxlogo}{}
\renewcommand{\releasename}{Release}
\makeindex

\makeatletter
\def\PYG@reset{\let\PYG@it=\relax \let\PYG@bf=\relax%
    \let\PYG@ul=\relax \let\PYG@tc=\relax%
    \let\PYG@bc=\relax \let\PYG@ff=\relax}
\def\PYG@tok#1{\csname PYG@tok@#1\endcsname}
\def\PYG@toks#1+{\ifx\relax#1\empty\else%
    \PYG@tok{#1}\expandafter\PYG@toks\fi}
\def\PYG@do#1{\PYG@bc{\PYG@tc{\PYG@ul{%
    \PYG@it{\PYG@bf{\PYG@ff{#1}}}}}}}
\def\PYG#1#2{\PYG@reset\PYG@toks#1+\relax+\PYG@do{#2}}

\expandafter\def\csname PYG@tok@gd\endcsname{\def\PYG@tc##1{\textcolor[rgb]{0.63,0.00,0.00}{##1}}}
\expandafter\def\csname PYG@tok@gu\endcsname{\let\PYG@bf=\textbf\def\PYG@tc##1{\textcolor[rgb]{0.50,0.00,0.50}{##1}}}
\expandafter\def\csname PYG@tok@gt\endcsname{\def\PYG@tc##1{\textcolor[rgb]{0.00,0.27,0.87}{##1}}}
\expandafter\def\csname PYG@tok@gs\endcsname{\let\PYG@bf=\textbf}
\expandafter\def\csname PYG@tok@gr\endcsname{\def\PYG@tc##1{\textcolor[rgb]{1.00,0.00,0.00}{##1}}}
\expandafter\def\csname PYG@tok@cm\endcsname{\let\PYG@it=\textit\def\PYG@tc##1{\textcolor[rgb]{0.25,0.50,0.56}{##1}}}
\expandafter\def\csname PYG@tok@vg\endcsname{\def\PYG@tc##1{\textcolor[rgb]{0.73,0.38,0.84}{##1}}}
\expandafter\def\csname PYG@tok@m\endcsname{\def\PYG@tc##1{\textcolor[rgb]{0.13,0.50,0.31}{##1}}}
\expandafter\def\csname PYG@tok@mh\endcsname{\def\PYG@tc##1{\textcolor[rgb]{0.13,0.50,0.31}{##1}}}
\expandafter\def\csname PYG@tok@cs\endcsname{\def\PYG@tc##1{\textcolor[rgb]{0.25,0.50,0.56}{##1}}\def\PYG@bc##1{\setlength{\fboxsep}{0pt}\colorbox[rgb]{1.00,0.94,0.94}{\strut ##1}}}
\expandafter\def\csname PYG@tok@ge\endcsname{\let\PYG@it=\textit}
\expandafter\def\csname PYG@tok@vc\endcsname{\def\PYG@tc##1{\textcolor[rgb]{0.73,0.38,0.84}{##1}}}
\expandafter\def\csname PYG@tok@il\endcsname{\def\PYG@tc##1{\textcolor[rgb]{0.13,0.50,0.31}{##1}}}
\expandafter\def\csname PYG@tok@go\endcsname{\def\PYG@tc##1{\textcolor[rgb]{0.20,0.20,0.20}{##1}}}
\expandafter\def\csname PYG@tok@cp\endcsname{\def\PYG@tc##1{\textcolor[rgb]{0.00,0.44,0.13}{##1}}}
\expandafter\def\csname PYG@tok@gi\endcsname{\def\PYG@tc##1{\textcolor[rgb]{0.00,0.63,0.00}{##1}}}
\expandafter\def\csname PYG@tok@gh\endcsname{\let\PYG@bf=\textbf\def\PYG@tc##1{\textcolor[rgb]{0.00,0.00,0.50}{##1}}}
\expandafter\def\csname PYG@tok@ni\endcsname{\let\PYG@bf=\textbf\def\PYG@tc##1{\textcolor[rgb]{0.84,0.33,0.22}{##1}}}
\expandafter\def\csname PYG@tok@nl\endcsname{\let\PYG@bf=\textbf\def\PYG@tc##1{\textcolor[rgb]{0.00,0.13,0.44}{##1}}}
\expandafter\def\csname PYG@tok@nn\endcsname{\let\PYG@bf=\textbf\def\PYG@tc##1{\textcolor[rgb]{0.05,0.52,0.71}{##1}}}
\expandafter\def\csname PYG@tok@no\endcsname{\def\PYG@tc##1{\textcolor[rgb]{0.38,0.68,0.84}{##1}}}
\expandafter\def\csname PYG@tok@na\endcsname{\def\PYG@tc##1{\textcolor[rgb]{0.25,0.44,0.63}{##1}}}
\expandafter\def\csname PYG@tok@nb\endcsname{\def\PYG@tc##1{\textcolor[rgb]{0.00,0.44,0.13}{##1}}}
\expandafter\def\csname PYG@tok@nc\endcsname{\let\PYG@bf=\textbf\def\PYG@tc##1{\textcolor[rgb]{0.05,0.52,0.71}{##1}}}
\expandafter\def\csname PYG@tok@nd\endcsname{\let\PYG@bf=\textbf\def\PYG@tc##1{\textcolor[rgb]{0.33,0.33,0.33}{##1}}}
\expandafter\def\csname PYG@tok@ne\endcsname{\def\PYG@tc##1{\textcolor[rgb]{0.00,0.44,0.13}{##1}}}
\expandafter\def\csname PYG@tok@nf\endcsname{\def\PYG@tc##1{\textcolor[rgb]{0.02,0.16,0.49}{##1}}}
\expandafter\def\csname PYG@tok@si\endcsname{\let\PYG@it=\textit\def\PYG@tc##1{\textcolor[rgb]{0.44,0.63,0.82}{##1}}}
\expandafter\def\csname PYG@tok@s2\endcsname{\def\PYG@tc##1{\textcolor[rgb]{0.25,0.44,0.63}{##1}}}
\expandafter\def\csname PYG@tok@vi\endcsname{\def\PYG@tc##1{\textcolor[rgb]{0.73,0.38,0.84}{##1}}}
\expandafter\def\csname PYG@tok@nt\endcsname{\let\PYG@bf=\textbf\def\PYG@tc##1{\textcolor[rgb]{0.02,0.16,0.45}{##1}}}
\expandafter\def\csname PYG@tok@nv\endcsname{\def\PYG@tc##1{\textcolor[rgb]{0.73,0.38,0.84}{##1}}}
\expandafter\def\csname PYG@tok@s1\endcsname{\def\PYG@tc##1{\textcolor[rgb]{0.25,0.44,0.63}{##1}}}
\expandafter\def\csname PYG@tok@gp\endcsname{\let\PYG@bf=\textbf\def\PYG@tc##1{\textcolor[rgb]{0.78,0.36,0.04}{##1}}}
\expandafter\def\csname PYG@tok@sh\endcsname{\def\PYG@tc##1{\textcolor[rgb]{0.25,0.44,0.63}{##1}}}
\expandafter\def\csname PYG@tok@ow\endcsname{\let\PYG@bf=\textbf\def\PYG@tc##1{\textcolor[rgb]{0.00,0.44,0.13}{##1}}}
\expandafter\def\csname PYG@tok@sx\endcsname{\def\PYG@tc##1{\textcolor[rgb]{0.78,0.36,0.04}{##1}}}
\expandafter\def\csname PYG@tok@bp\endcsname{\def\PYG@tc##1{\textcolor[rgb]{0.00,0.44,0.13}{##1}}}
\expandafter\def\csname PYG@tok@c1\endcsname{\let\PYG@it=\textit\def\PYG@tc##1{\textcolor[rgb]{0.25,0.50,0.56}{##1}}}
\expandafter\def\csname PYG@tok@kc\endcsname{\let\PYG@bf=\textbf\def\PYG@tc##1{\textcolor[rgb]{0.00,0.44,0.13}{##1}}}
\expandafter\def\csname PYG@tok@c\endcsname{\let\PYG@it=\textit\def\PYG@tc##1{\textcolor[rgb]{0.25,0.50,0.56}{##1}}}
\expandafter\def\csname PYG@tok@mf\endcsname{\def\PYG@tc##1{\textcolor[rgb]{0.13,0.50,0.31}{##1}}}
\expandafter\def\csname PYG@tok@err\endcsname{\def\PYG@bc##1{\setlength{\fboxsep}{0pt}\fcolorbox[rgb]{1.00,0.00,0.00}{1,1,1}{\strut ##1}}}
\expandafter\def\csname PYG@tok@kd\endcsname{\let\PYG@bf=\textbf\def\PYG@tc##1{\textcolor[rgb]{0.00,0.44,0.13}{##1}}}
\expandafter\def\csname PYG@tok@ss\endcsname{\def\PYG@tc##1{\textcolor[rgb]{0.32,0.47,0.09}{##1}}}
\expandafter\def\csname PYG@tok@sr\endcsname{\def\PYG@tc##1{\textcolor[rgb]{0.14,0.33,0.53}{##1}}}
\expandafter\def\csname PYG@tok@mo\endcsname{\def\PYG@tc##1{\textcolor[rgb]{0.13,0.50,0.31}{##1}}}
\expandafter\def\csname PYG@tok@mi\endcsname{\def\PYG@tc##1{\textcolor[rgb]{0.13,0.50,0.31}{##1}}}
\expandafter\def\csname PYG@tok@kn\endcsname{\let\PYG@bf=\textbf\def\PYG@tc##1{\textcolor[rgb]{0.00,0.44,0.13}{##1}}}
\expandafter\def\csname PYG@tok@o\endcsname{\def\PYG@tc##1{\textcolor[rgb]{0.40,0.40,0.40}{##1}}}
\expandafter\def\csname PYG@tok@kr\endcsname{\let\PYG@bf=\textbf\def\PYG@tc##1{\textcolor[rgb]{0.00,0.44,0.13}{##1}}}
\expandafter\def\csname PYG@tok@s\endcsname{\def\PYG@tc##1{\textcolor[rgb]{0.25,0.44,0.63}{##1}}}
\expandafter\def\csname PYG@tok@kp\endcsname{\def\PYG@tc##1{\textcolor[rgb]{0.00,0.44,0.13}{##1}}}
\expandafter\def\csname PYG@tok@w\endcsname{\def\PYG@tc##1{\textcolor[rgb]{0.73,0.73,0.73}{##1}}}
\expandafter\def\csname PYG@tok@kt\endcsname{\def\PYG@tc##1{\textcolor[rgb]{0.56,0.13,0.00}{##1}}}
\expandafter\def\csname PYG@tok@sc\endcsname{\def\PYG@tc##1{\textcolor[rgb]{0.25,0.44,0.63}{##1}}}
\expandafter\def\csname PYG@tok@sb\endcsname{\def\PYG@tc##1{\textcolor[rgb]{0.25,0.44,0.63}{##1}}}
\expandafter\def\csname PYG@tok@k\endcsname{\let\PYG@bf=\textbf\def\PYG@tc##1{\textcolor[rgb]{0.00,0.44,0.13}{##1}}}
\expandafter\def\csname PYG@tok@se\endcsname{\let\PYG@bf=\textbf\def\PYG@tc##1{\textcolor[rgb]{0.25,0.44,0.63}{##1}}}
\expandafter\def\csname PYG@tok@sd\endcsname{\let\PYG@it=\textit\def\PYG@tc##1{\textcolor[rgb]{0.25,0.44,0.63}{##1}}}

\def\PYGZbs{\char`\\}
\def\PYGZus{\char`\_}
\def\PYGZob{\char`\{}
\def\PYGZcb{\char`\}}
\def\PYGZca{\char`\^}
\def\PYGZam{\char`\&}
\def\PYGZlt{\char`\<}
\def\PYGZgt{\char`\>}
\def\PYGZsh{\char`\#}
\def\PYGZpc{\char`\%}
\def\PYGZdl{\char`\$}
\def\PYGZhy{\char`\-}
\def\PYGZsq{\char`\'}
\def\PYGZdq{\char`\"}
\def\PYGZti{\char`\~}
% for compatibility with earlier versions
\def\PYGZat{@}
\def\PYGZlb{[}
\def\PYGZrb{]}
\makeatother

\begin{document}
\shorthandoff{"}
\maketitle
\tableofcontents
\phantomsection\label{index::doc}


Das Python-Paket \textbf{pyzufall} generiert zufällige Daten wie Namen, Fantasieworte, Personen oder Sätze.

Es wurde für \href{http://satzgenerator.de/}{satzgenerator.de} entwickelt, kann aber vielfältig eingesetzt werden.

Pyzufall ist Open Source, der Quelltext kann auf \href{https://github.com/davidak/pyzufall/}{github.com} runtergeladen werden.


\chapter{Entstehung}
\label{entstehung::doc}\label{entstehung:entstehung}\label{entstehung:dokumentation-von-pyzufall}

\section{Das Spiel}
\label{entstehung:das-spiel}
Als Kind hatte ich bei meiner Oma öfter das Spiel \href{http://www.mama-tipps.de/tipp/Opa-plaetschert-Badewanne.html}{Opa plätschert lustig in der Badewanne} gespielt.

Dabei hat jeder Spieler ein Blatt Papier, dass er quer nimmt und als erstes einen Namen oder eine Person darauf schreibt, es an der Stelle faltet, so dass man es nicht mehr lesen kann und es im Uhrzeigersinn weitergibt. Dann schreibt jeder auf das erhaltene Blatt ein Verb, gibt es weiter und schribt ein Adjektiv und nach nochmaligem Weitergeben einen Ort darauf.

So entstehen absurde und zufällige Sätze. Das war immer sehr witzig.


\section{Das Programm}
\label{entstehung:das-programm}
Als ich älter wurde, begann ich Programmieren zu lernen. Dabei hat mich immer begeistert, durch Zufall etwas zu generieren, was teileise einen Sinn ergibt, aber oft sehr absurd und dadurch lustig ist.

So ist ein \href{http://davidak.de/wiki/perl/personendatengenerator}{Personendatengenerator} entstanden, mit dem eine \href{http://davidak.de/personen/}{Personendatenbank} befüllt wurde.

Auch hab ich ein Script geschrieben, dass Sätze nach dem Muster des Spiels generiert. Erst in Perl und dann in Python. Diese Sätze werden natürlich auf Dauer langweilig.

Da mittlerweile Python die Programmiersprache meiner Wahl ist, habe ich das Script darin weiterentwickelt, mit diversen Satz-Schemata und anderen tollen Funktionen.


\section{Der Satzgenerator}
\label{entstehung:der-satzgenerator}
Inzwischen gibt es \href{http://satzgenerator.de/}{satzgenerator.de}. Auf dieser Webseite werden zufällige Sätze generiert, die bewertet und geteilt werden können.

Für die Generierung der Sätze wird das Python-Modul \href{https://github.com/davidak/pyzufall}{pyzufall} genutzt. Die Seite ist auch in Python programmiert, benutzt das Web-Framework \href{http://bottlepy.org/}{Bottle} und eine MySQL-Datenbank für die Speicherung der Sätze und Bewertungen.

Pyzufall ist Open Source und wird bereits für {\hyperref[benutzer::doc]{\emph{andere Projekte}}} benutzt.


\chapter{Installation}
\label{installation:installation}\label{installation::doc}
Die aktuellste Version findest du auf \href{https://github.com/davidak/pyzufall}{github}. Eine stabile Version wird es auf \href{https://pypi.python.org/}{PyPI} geben.


\section{Manuelle Installation}
\label{installation:manuelle-installation}
Klone das git-Repository in dein Projekt-Verzeichnis:

\begin{Verbatim}[commandchars=\\\{\}]
\$ mkdir meinprojekt
\$ cd meinprojekt
\$ git clone https://github.com/davidak/pyzufall.git
\end{Verbatim}


\section{Unittests}
\label{installation:unittests}
Um den Code auf deinem System zu testen, führe folgenden Befehl im Projekt-Verzeichnis aus:

\begin{Verbatim}[commandchars=\\\{\}]
\$ nosetests -v
test.test\_satz ... ok
test.test\_satz\_frage ... ok
test.test\_test ... ok
test.test\_lese ... ok
test.test\_erste\_gross ... ok

----------------------------------------------------------------------
Ran 5 tests in 0.807s

OK
\end{Verbatim}

Für die Unittests wird \href{https://nose.readthedocs.org/en/latest/}{nose} verwendet und muss installiert sein.


\chapter{Verwenden}
\label{verwenden::doc}\label{verwenden:verwenden}
Anhand von Beispielen werden die verschiedenen Module von \textbf{pyzufall} vorgestellt.


\section{Name generieren}
\label{verwenden:name-generieren}
Um einen Namen zu generieren wird das Modul {\hyperref[module:module-pyzufall.generator]{\code{pyzufall.generator}}} verwendet, dass eine Vielzahl von Funktionen für das generieren von Daten bereitstellt.

\begin{Verbatim}[commandchars=\\\{\}]
\PYG{k+kn}{from} \PYG{n+nn}{pyzufall.generator} \PYG{k+kn}{import} \PYG{n}{vorname\PYGZus{}m}\PYG{p}{,} \PYG{n}{nachname}
\PYG{n}{name} \PYG{o}{=} \PYG{n}{vorname\PYGZus{}m}\PYG{p}{(}\PYG{p}{)} \PYG{o}{+} \PYG{l+s}{\PYGZsq{}}\PYG{l+s}{ }\PYG{l+s}{\PYGZsq{}} \PYG{o}{+} \PYG{n}{nachname}\PYG{p}{(}\PYG{p}{)}
\PYG{k}{print}\PYG{p}{(}\PYG{n}{name}\PYG{p}{)}
\end{Verbatim}


\strong{Siehe auch:}


Eine Übersicht aller Funktionen findest du in der {\hyperref[module::doc]{\emph{Referenz}}}.




\section{Satz generieren}
\label{verwenden:satz-generieren}
Das Modul {\hyperref[module:module-pyzufall.satz]{\code{pyzufall.satz}}} generiert zufällige Sätze nach diversen Schemata.

\begin{Verbatim}[commandchars=\\\{\}]
\PYG{k+kn}{from} \PYG{n+nn}{pyzufall.satz} \PYG{k+kn}{import} \PYG{n}{satz}
\PYG{n}{satz} \PYG{o}{=} \PYG{n}{satz}\PYG{p}{(}\PYG{p}{)}
\PYG{k}{print}\PYG{p}{(}\PYG{n}{satz}\PYG{p}{)}
\end{Verbatim}


\section{Person generieren}
\label{verwenden:person-generieren}
Die Klasse {\hyperref[module:pyzufall.person.Person]{\code{Person}}} des Moduls {\hyperref[module:module-pyzufall.person]{\code{pyzufall.person}}} generiert eine Person mit diversen Daten. Es können alle Daten aufeinmal ausgegeben werden oder einzeln auf diese zugegriffen werden.

\begin{Verbatim}[commandchars=\\\{\}]
\PYG{k+kn}{from} \PYG{n+nn}{pyzufall.person} \PYG{k+kn}{import} \PYG{n}{Person}
\PYG{n}{p1} \PYG{o}{=} \PYG{n}{Person}\PYG{p}{(}\PYG{p}{)}
\PYG{n}{p2} \PYG{o}{=} \PYG{n}{Person}\PYG{p}{(}\PYG{p}{)}
\PYG{k}{print}\PYG{p}{(}\PYG{n}{p1}\PYG{p}{)}
\PYG{k}{print}\PYG{p}{(}\PYG{n}{p1}\PYG{o}{.}\PYG{n}{vorname} \PYG{o}{+} \PYG{l+s}{\PYGZdq{}}\PYG{l+s}{ und }\PYG{l+s}{\PYGZdq{}} \PYG{o}{+} \PYG{n}{p2}\PYG{o}{.}\PYG{n}{vorname} \PYG{o}{+} \PYG{l+s}{\PYGZdq{}}\PYG{l+s}{ sitzen auf einer Bank im Park.}\PYG{l+s+se}{\PYGZbs{}n}\PYG{l+s}{\PYGZdq{}}\PYG{p}{)}
\PYG{k}{print}\PYG{p}{(}\PYG{n}{p2}\PYG{o}{.}\PYG{n}{vorname} \PYG{o}{+} \PYG{l+s}{\PYGZdq{}}\PYG{l+s}{ (}\PYG{l+s}{\PYGZdq{}} \PYG{o}{+} \PYG{n+nb}{str}\PYG{p}{(}\PYG{n}{p2}\PYG{o}{.}\PYG{n}{alter}\PYG{p}{)} \PYG{o}{+} \PYG{l+s}{\PYGZdq{}}\PYG{l+s}{) wohnt in }\PYG{l+s}{\PYGZdq{}} \PYG{o}{+} \PYG{n}{p2}\PYG{o}{.}\PYG{n}{wohnort} \PYG{o}{+} \PYG{l+s}{\PYGZdq{}}\PYG{l+s}{.}\PYG{l+s}{\PYGZdq{}}\PYG{p}{)}
\end{Verbatim}

\begin{Verbatim}[commandchars=\\\{\}]
********************************************************************************
Name: Kornelia Eismann
Geschlecht: weiblich
Geburtsdatum: 27.04.1974 (39)
Geburtsort: Geesthacht
Wohnort: Halle
Beruf: Technische Zeichnerin
Interessen: Würfelspiele
Lieblingsfarbe: Weiß
Lieblingsessen: Steak
Motto: Augen auf beim Eierkauf.
********************************************************************************

Kornelia und Thorge sitzen auf einer Bank im Park.

Thorge (58) wohnt in Stadthagen.
\end{Verbatim}


\chapter{Beitragen}
\label{beitragen:beitragen}\label{beitragen::doc}
Du kannst bei diesem Open Source-Projekt mitwirken, indem du \href{https://github.com/davidak/pyzufall/issues/}{Fehler berichtest}, neue Datensätze hinzufügst oder sogar mitprogrammierst.

Die Vielfalt und Anzahl der möglichen Sätze steigt mit den Datensätzen. An einer einfachen Möglichkeit, Daten hinzuzufügen, wird gearbeitet.

\begin{notice}{note}{Zu tun}

Dokuwiki auf satzgenerator.de/beitragen einrichten mit Kopie der Datensätze. Bearbeiten nach Registrierung möglich.
\end{notice}

Pull-Requests auf github sind willkommen.


\chapter{Benutzer}
\label{benutzer::doc}\label{benutzer:benutzer}
Hier ist eine Liste mit Projekten, die pyzufall verwenden:
\begin{itemize}
\item {} 
\href{http://satzgenerator.de/}{satzgenerator.de}

\item {} 
\href{https://github.com/davidak/python-random-vcard-generator}{Python Random VCard-Generator}

\end{itemize}

Dein Projekt füge ich auch gerne hinzu.

Einfach eine E-Mail mit Beschreibung und Link an post at davidak punkt de oder das \href{http://davidak.de/kontakt}{Kontaktformular} benutzen.


\chapter{Aufgaben}
\label{todo::doc}\label{todo:aufgaben}
Für Fehlerberichte und Feature-Requests wird der \href{https://github.com/davidak/pyzufall/issues}{Bugtracker auf github} benutzt.

Auch im Quelltext gibt es Hinweise auf nötige Anpassungen:

\begin{notice}{note}{Zu tun}

Dokuwiki auf satzgenerator.de/beitragen einrichten mit Kopie der Datensätze. Bearbeiten nach Registrierung möglich.
\end{notice}

(Der {\hyperref[beitragen:index-0]{\emph{ursprüngliche Eintrag}}} steht in /Users/davidak/BTSync/code/pyzufall/doc/beitragen.rst, Zeile 8.)

\begin{notice}{note}{Zu tun}

Funktion programmieren
\end{notice}

(Der {\hyperref[module:index-0]{\emph{ursprüngliche Eintrag}}} steht in /Users/davidak/BTSync/code/pyzufall/pyzufall/generator.py:docstring of pyzufall.generator.firma, Zeile 3.)

\begin{notice}{note}{Zu tun}

''...''ste nicht immer richtig:
\begin{itemize}
\item {} 
Bruder Dennis war der monströsste Mönch im Kloster.

\item {} 
Schwester Lara ist die diskretste Nonne in der Abtei.

\item {} 
Bruder Marcel war der gerechtste Mönch im Orden.

\item {} 
Bruder Nicolaus ist der falschste Mönch im Kloster.

\end{itemize}
\end{notice}

(Der {\hyperref[module:index-1]{\emph{ursprüngliche Eintrag}}} steht in /Users/davidak/BTSync/code/pyzufall/pyzufall/satz.py:docstring of pyzufall.satz.satz\_kloster, Zeile 5.)


\chapter{Referenz der Module und ihrer Funktionen}
\label{module::doc}\label{module:referenz-der-module-und-ihrer-funktionen}
Die Module und ihre Funktionen.


\section{pyzufall.helfer}
\label{module:pyzufall-helfer}\label{module:module-pyzufall.helfer}\index{pyzufall.helfer (Modul)}\index{e16() (in Modul pyzufall.helfer)}

\begin{fulllineitems}
\phantomsection\label{module:pyzufall.helfer.e16}\pysiglinewithargsret{\code{pyzufall.helfer.}\bfcode{e16}}{\emph{wert}}{}
Der übergebene Wert wird mit einer Wahrscheinlichkeit von 16\% zurückgegeben.

\end{fulllineitems}

\index{e25() (in Modul pyzufall.helfer)}

\begin{fulllineitems}
\phantomsection\label{module:pyzufall.helfer.e25}\pysiglinewithargsret{\code{pyzufall.helfer.}\bfcode{e25}}{\emph{wert}}{}
Der übergebene Wert wird mit einer Wahrscheinlichkeit von 25\% zurückgegeben.

\end{fulllineitems}

\index{e50() (in Modul pyzufall.helfer)}

\begin{fulllineitems}
\phantomsection\label{module:pyzufall.helfer.e50}\pysiglinewithargsret{\code{pyzufall.helfer.}\bfcode{e50}}{\emph{wert}}{}
Der übergebene Wert wird mit einer Wahrscheinlichkeit von 50\% zurückgegeben.

\end{fulllineitems}

\index{e75() (in Modul pyzufall.helfer)}

\begin{fulllineitems}
\phantomsection\label{module:pyzufall.helfer.e75}\pysiglinewithargsret{\code{pyzufall.helfer.}\bfcode{e75}}{\emph{wert}}{}
Der übergebene Wert wird mit einer Wahrscheinlichkeit von 75\% zurückgegeben.

\end{fulllineitems}

\index{erste\_gross() (in Modul pyzufall.helfer)}

\begin{fulllineitems}
\phantomsection\label{module:pyzufall.helfer.erste_gross}\pysiglinewithargsret{\code{pyzufall.helfer.}\bfcode{erste\_gross}}{\emph{s}}{}
Macht den ersten Buchstaben gross.

\end{fulllineitems}

\index{lese() (in Modul pyzufall.helfer)}

\begin{fulllineitems}
\phantomsection\label{module:pyzufall.helfer.lese}\pysiglinewithargsret{\code{pyzufall.helfer.}\bfcode{lese}}{\emph{dateiname}}{}
Liest die Datei mit dem übergebenen Namen aus data/ zeilenweise ein und gib eine Liste zurück.

\href{http://stackoverflow.com/questions/10174211/make-an-always-relative-to-current-module-file-path}{http://stackoverflow.com/questions/10174211/make-an-always-relative-to-current-module-file-path}

\end{fulllineitems}



\section{pyzufall.generator}
\label{module:pyzufall-generator}\label{module:module-pyzufall.generator}\index{pyzufall.generator (Modul)}\index{adjektiv() (in Modul pyzufall.generator)}

\begin{fulllineitems}
\phantomsection\label{module:pyzufall.generator.adjektiv}\pysiglinewithargsret{\code{pyzufall.generator.}\bfcode{adjektiv}}{}{}
Gibt ein Adjektiv zurück.

\end{fulllineitems}

\index{band() (in Modul pyzufall.generator)}

\begin{fulllineitems}
\phantomsection\label{module:pyzufall.generator.band}\pysiglinewithargsret{\code{pyzufall.generator.}\bfcode{band}}{}{}
Gibt einen fiktiven Bandnamen zurück.

\end{fulllineitems}

\index{bandart() (in Modul pyzufall.generator)}

\begin{fulllineitems}
\phantomsection\label{module:pyzufall.generator.bandart}\pysiglinewithargsret{\code{pyzufall.generator.}\bfcode{bandart}}{}{}
Gibt eine Bandart zurück.

Beispiel: `Gothic Metal Band'

\end{fulllineitems}

\index{baum() (in Modul pyzufall.generator)}

\begin{fulllineitems}
\phantomsection\label{module:pyzufall.generator.baum}\pysiglinewithargsret{\code{pyzufall.generator.}\bfcode{baum}}{}{}
Gibt einen Baum zurück.

\end{fulllineitems}

\index{beilage() (in Modul pyzufall.generator)}

\begin{fulllineitems}
\phantomsection\label{module:pyzufall.generator.beilage}\pysiglinewithargsret{\code{pyzufall.generator.}\bfcode{beilage}}{}{}
Gibt eine Beilage zum Essen zurück.

\end{fulllineitems}

\index{beruf\_m() (in Modul pyzufall.generator)}

\begin{fulllineitems}
\phantomsection\label{module:pyzufall.generator.beruf_m}\pysiglinewithargsret{\code{pyzufall.generator.}\bfcode{beruf\_m}}{}{}
Gibt eine männliche Berufsbezeichnung zurück.

\end{fulllineitems}

\index{beruf\_w() (in Modul pyzufall.generator)}

\begin{fulllineitems}
\phantomsection\label{module:pyzufall.generator.beruf_w}\pysiglinewithargsret{\code{pyzufall.generator.}\bfcode{beruf\_w}}{}{}
Gibt eine weibliche Berufsbezeichnung zurück.

\end{fulllineitems}

\index{color() (in Modul pyzufall.generator)}

\begin{fulllineitems}
\phantomsection\label{module:pyzufall.generator.color}\pysiglinewithargsret{\code{pyzufall.generator.}\bfcode{color}}{}{}
Gibt eine Farbe auf englisch zurück.

\end{fulllineitems}

\index{datum() (in Modul pyzufall.generator)}

\begin{fulllineitems}
\phantomsection\label{module:pyzufall.generator.datum}\pysiglinewithargsret{\code{pyzufall.generator.}\bfcode{datum}}{}{}
Gibt ein gültiges Datum zwischen dem 01.01.1950 und 31.12.2012 zurück.

\end{fulllineitems}

\index{essen() (in Modul pyzufall.generator)}

\begin{fulllineitems}
\phantomsection\label{module:pyzufall.generator.essen}\pysiglinewithargsret{\code{pyzufall.generator.}\bfcode{essen}}{}{}
Gibt ein Gericht zurück.

\end{fulllineitems}

\index{farbe() (in Modul pyzufall.generator)}

\begin{fulllineitems}
\phantomsection\label{module:pyzufall.generator.farbe}\pysiglinewithargsret{\code{pyzufall.generator.}\bfcode{farbe}}{}{}
Gibt eine Farbe zurück.

\end{fulllineitems}

\index{firma() (in Modul pyzufall.generator)}

\begin{fulllineitems}
\phantomsection\label{module:pyzufall.generator.firma}\pysiglinewithargsret{\code{pyzufall.generator.}\bfcode{firma}}{}{}
Gibt einen fiktiven Firmenname zurück.

\begin{notice}{note}{Zu tun}

Funktion programmieren
\end{notice}

\end{fulllineitems}

\index{geburtsdatum() (in Modul pyzufall.generator)}

\begin{fulllineitems}
\phantomsection\label{module:pyzufall.generator.geburtsdatum}\pysiglinewithargsret{\code{pyzufall.generator.}\bfcode{geburtsdatum}}{}{}
Gibt ein gültiges Datum zwischen dem 01.01.1910 und 31.12.2012 zurück.

\end{fulllineitems}

\index{gegenstand() (in Modul pyzufall.generator)}

\begin{fulllineitems}
\phantomsection\label{module:pyzufall.generator.gegenstand}\pysiglinewithargsret{\code{pyzufall.generator.}\bfcode{gegenstand}}{}{}
Gibt einen Gegenstand zurück.

\end{fulllineitems}

\index{geschlecht() (in Modul pyzufall.generator)}

\begin{fulllineitems}
\phantomsection\label{module:pyzufall.generator.geschlecht}\pysiglinewithargsret{\code{pyzufall.generator.}\bfcode{geschlecht}}{}{}
Gibt ein zufälliges Geschlecht zurück.

1 = männlich
0 = weiblich

2011 gibt es laut Statistik 51,18\% weibliche Personen in Deutschland:
\href{https://www.destatis.de/DE/ZahlenFakten/GesellschaftStaat/Bevoelkerung/Bevoelkerungsstand/Tabellen/Zensus\_Geschlecht\_Staatsangehoerigkeit.html}{https://www.destatis.de/DE/ZahlenFakten/GesellschaftStaat/Bevoelkerung/Bevoelkerungsstand/Tabellen/Zensus\_Geschlecht\_Staatsangehoerigkeit.html}

\end{fulllineitems}

\index{interesse() (in Modul pyzufall.generator)}

\begin{fulllineitems}
\phantomsection\label{module:pyzufall.generator.interesse}\pysiglinewithargsret{\code{pyzufall.generator.}\bfcode{interesse}}{}{}
Gibt ein zufälliges Interesse bzw Hobby zurück.

\end{fulllineitems}

\index{koerperteil() (in Modul pyzufall.generator)}

\begin{fulllineitems}
\phantomsection\label{module:pyzufall.generator.koerperteil}\pysiglinewithargsret{\code{pyzufall.generator.}\bfcode{koerperteil}}{}{}
Gibt ein Körperteil zurück.

\end{fulllineitems}

\index{nachname() (in Modul pyzufall.generator)}

\begin{fulllineitems}
\phantomsection\label{module:pyzufall.generator.nachname}\pysiglinewithargsret{\code{pyzufall.generator.}\bfcode{nachname}}{}{}
Gibt einen Nachnamen zurück.

\end{fulllineitems}

\index{objekt() (in Modul pyzufall.generator)}

\begin{fulllineitems}
\phantomsection\label{module:pyzufall.generator.objekt}\pysiglinewithargsret{\code{pyzufall.generator.}\bfcode{objekt}}{}{}
Gibt ein Objekt zurück.

\end{fulllineitems}

\index{objekt\_m() (in Modul pyzufall.generator)}

\begin{fulllineitems}
\phantomsection\label{module:pyzufall.generator.objekt_m}\pysiglinewithargsret{\code{pyzufall.generator.}\bfcode{objekt\_m}}{\emph{s}}{}
Bringt ein Objekt in Berzug zu einer männlichen Person.

Beispiel:
`der Bär' wird zu `den Bären' oder `seinen Bären'

\end{fulllineitems}

\index{objekt\_w() (in Modul pyzufall.generator)}

\begin{fulllineitems}
\phantomsection\label{module:pyzufall.generator.objekt_w}\pysiglinewithargsret{\code{pyzufall.generator.}\bfcode{objekt\_w}}{\emph{s}}{}
Bringt ein Objekt in Berzug zu einer weiblichen Person.

Beispiel:
`der Bär' wird zu `den Bären' oder `ihren Bären'

\end{fulllineitems}

\index{ort() (in Modul pyzufall.generator)}

\begin{fulllineitems}
\phantomsection\label{module:pyzufall.generator.ort}\pysiglinewithargsret{\code{pyzufall.generator.}\bfcode{ort}}{}{}
Gibt eine Ortsangabe zurück.

Beispiel: `im Flur'

\end{fulllineitems}

\index{person() (in Modul pyzufall.generator)}

\begin{fulllineitems}
\phantomsection\label{module:pyzufall.generator.person}\pysiglinewithargsret{\code{pyzufall.generator.}\bfcode{person}}{}{}
Gibt eine zufällige Person zurück.

\end{fulllineitems}

\index{person\_m() (in Modul pyzufall.generator)}

\begin{fulllineitems}
\phantomsection\label{module:pyzufall.generator.person_m}\pysiglinewithargsret{\code{pyzufall.generator.}\bfcode{person\_m}}{}{}
Gibt eine männliche Person zurück.

\end{fulllineitems}

\index{person\_objekt\_m() (in Modul pyzufall.generator)}

\begin{fulllineitems}
\phantomsection\label{module:pyzufall.generator.person_objekt_m}\pysiglinewithargsret{\code{pyzufall.generator.}\bfcode{person\_objekt\_m}}{}{}
Gibt eine Person als Objekt in Bezug auf eine männliche Person zurück.

Beispiel: seine Mitarbeiterin

\end{fulllineitems}

\index{person\_objekt\_w() (in Modul pyzufall.generator)}

\begin{fulllineitems}
\phantomsection\label{module:pyzufall.generator.person_objekt_w}\pysiglinewithargsret{\code{pyzufall.generator.}\bfcode{person\_objekt\_w}}{}{}
Gibt eine Person als Objekt in Bezug auf eine weibliche Person zurück.

Beispiel: ihre Mutter

\end{fulllineitems}

\index{person\_w() (in Modul pyzufall.generator)}

\begin{fulllineitems}
\phantomsection\label{module:pyzufall.generator.person_w}\pysiglinewithargsret{\code{pyzufall.generator.}\bfcode{person\_w}}{}{}
Gibt eine weibliche Person zurück.

\end{fulllineitems}

\index{pflanze() (in Modul pyzufall.generator)}

\begin{fulllineitems}
\phantomsection\label{module:pyzufall.generator.pflanze}\pysiglinewithargsret{\code{pyzufall.generator.}\bfcode{pflanze}}{}{}
Gibt eine Pflanze zurück.

\end{fulllineitems}

\index{sprichwort() (in Modul pyzufall.generator)}

\begin{fulllineitems}
\phantomsection\label{module:pyzufall.generator.sprichwort}\pysiglinewithargsret{\code{pyzufall.generator.}\bfcode{sprichwort}}{}{}
Gibt ein Sprichwort zurück.

\end{fulllineitems}

\index{stadt() (in Modul pyzufall.generator)}

\begin{fulllineitems}
\phantomsection\label{module:pyzufall.generator.stadt}\pysiglinewithargsret{\code{pyzufall.generator.}\bfcode{stadt}}{}{}
Gibt eine Stadt zurück.

\end{fulllineitems}

\index{stadt\_bl() (in Modul pyzufall.generator)}

\begin{fulllineitems}
\phantomsection\label{module:pyzufall.generator.stadt_bl}\pysiglinewithargsret{\code{pyzufall.generator.}\bfcode{stadt\_bl}}{}{}
Gibt eine Stadt mit Bundesland zurück.

\end{fulllineitems}

\index{tier() (in Modul pyzufall.generator)}

\begin{fulllineitems}
\phantomsection\label{module:pyzufall.generator.tier}\pysiglinewithargsret{\code{pyzufall.generator.}\bfcode{tier}}{}{}
Gibt ein Tier zurück.

\end{fulllineitems}

\index{trinken() (in Modul pyzufall.generator)}

\begin{fulllineitems}
\phantomsection\label{module:pyzufall.generator.trinken}\pysiglinewithargsret{\code{pyzufall.generator.}\bfcode{trinken}}{}{}
Gibt ein Getränk zurück.

\end{fulllineitems}

\index{verbd() (in Modul pyzufall.generator)}

\begin{fulllineitems}
\phantomsection\label{module:pyzufall.generator.verbd}\pysiglinewithargsret{\code{pyzufall.generator.}\bfcode{verbd}}{}{}
Gibt ein ditransitives Verb zurück.

\href{http://de.wikipedia.org/wiki/Transitivität\_(Grammatik)\#Festlegung\_der\_Transitivit.C3.A4t\_eines\_Verbs/}{Beschreibung auf Wikipedia}

\end{fulllineitems}

\index{verbi() (in Modul pyzufall.generator)}

\begin{fulllineitems}
\phantomsection\label{module:pyzufall.generator.verbi}\pysiglinewithargsret{\code{pyzufall.generator.}\bfcode{verbi}}{}{}
Gibt ein intransitives Verb zurück.

\href{http://de.wikipedia.org/wiki/Transitivität\_(Grammatik)\#Festlegung\_der\_Transitivit.C3.A4t\_eines\_Verbs/}{Beschreibung auf Wikipedia}

\end{fulllineitems}

\index{verbi2() (in Modul pyzufall.generator)}

\begin{fulllineitems}
\phantomsection\label{module:pyzufall.generator.verbi2}\pysiglinewithargsret{\code{pyzufall.generator.}\bfcode{verbi2}}{}{}
Gibt ein intransitives, getrenntes Verb zurück.

\href{http://de.wikipedia.org/wiki/Transitivität\_(Grammatik)\#Festlegung\_der\_Transitivit.C3.A4t\_eines\_Verbs/}{Beschreibung auf Wikipedia}

\end{fulllineitems}

\index{verbn() (in Modul pyzufall.generator)}

\begin{fulllineitems}
\phantomsection\label{module:pyzufall.generator.verbn}\pysiglinewithargsret{\code{pyzufall.generator.}\bfcode{verbn}}{}{}
Gibt ein nullwertiges Verb zurück.

\href{http://de.wikipedia.org/wiki/Transitivität\_(Grammatik)\#Festlegung\_der\_Transitivit.C3.A4t\_eines\_Verbs/}{Beschreibung auf Wikipedia}

\end{fulllineitems}

\index{verbt() (in Modul pyzufall.generator)}

\begin{fulllineitems}
\phantomsection\label{module:pyzufall.generator.verbt}\pysiglinewithargsret{\code{pyzufall.generator.}\bfcode{verbt}}{}{}
Gibt ein transitives Verb zurück.

\href{http://de.wikipedia.org/wiki/Transitivität\_(Grammatik)\#Festlegung\_der\_Transitivit.C3.A4t\_eines\_Verbs/}{Beschreibung auf Wikipedia}

\end{fulllineitems}

\index{verbt2() (in Modul pyzufall.generator)}

\begin{fulllineitems}
\phantomsection\label{module:pyzufall.generator.verbt2}\pysiglinewithargsret{\code{pyzufall.generator.}\bfcode{verbt2}}{}{}
Gibt ein intransitives, getrenntes Verb zurück.

\href{http://de.wikipedia.org/wiki/Transitivität\_(Grammatik)\#Festlegung\_der\_Transitivit.C3.A4t\_eines\_Verbs/}{Beschreibung auf Wikipedia}

\end{fulllineitems}

\index{vorname() (in Modul pyzufall.generator)}

\begin{fulllineitems}
\phantomsection\label{module:pyzufall.generator.vorname}\pysiglinewithargsret{\code{pyzufall.generator.}\bfcode{vorname}}{}{}
Gibt einen zufälligen Vornamen zurück.

\end{fulllineitems}

\index{vorname\_m() (in Modul pyzufall.generator)}

\begin{fulllineitems}
\phantomsection\label{module:pyzufall.generator.vorname_m}\pysiglinewithargsret{\code{pyzufall.generator.}\bfcode{vorname\_m}}{}{}
Gibt einen männlichen Vornamen zurück.

\end{fulllineitems}

\index{vorname\_w() (in Modul pyzufall.generator)}

\begin{fulllineitems}
\phantomsection\label{module:pyzufall.generator.vorname_w}\pysiglinewithargsret{\code{pyzufall.generator.}\bfcode{vorname\_w}}{}{}
Gibt einen weiblichen Vornamen zurück.

\end{fulllineitems}

\index{wort() (in Modul pyzufall.generator)}

\begin{fulllineitems}
\phantomsection\label{module:pyzufall.generator.wort}\pysiglinewithargsret{\code{pyzufall.generator.}\bfcode{wort}}{}{}
Gibt ein Fantasiewort zurück.

\end{fulllineitems}

\index{zahl() (in Modul pyzufall.generator)}

\begin{fulllineitems}
\phantomsection\label{module:pyzufall.generator.zahl}\pysiglinewithargsret{\code{pyzufall.generator.}\bfcode{zahl}}{}{}
Gibt eine Zahl zwischen 0 und 100 zurück.

\end{fulllineitems}



\section{pyzufall.person}
\label{module:module-pyzufall.person}\label{module:pyzufall-person}\index{pyzufall.person (Modul)}\index{Person (Klasse in pyzufall.person)}

\begin{fulllineitems}
\phantomsection\label{module:pyzufall.person.Person}\pysigline{\strong{class }\code{pyzufall.person.}\bfcode{Person}}
Generiert Daten einer zufälligen und fiktiven Person.

\DUspan{}{Neu in Version 0.9.}

\end{fulllineitems}



\section{pyzufall.satz}
\label{module:module-pyzufall.satz}\label{module:pyzufall-satz}\index{pyzufall.satz (Modul)}\index{satz() (in Modul pyzufall.satz)}

\begin{fulllineitems}
\phantomsection\label{module:pyzufall.satz.satz}\pysiglinewithargsret{\code{pyzufall.satz.}\bfcode{satz}}{}{}
Generiert einen zufälligen Satz.

20\% Standard-Sätze, 20\% Fragen und 60\% Themen-Sätze

\end{fulllineitems}

\index{satz\_absurde\_farbfunktion() (in Modul pyzufall.satz)}

\begin{fulllineitems}
\phantomsection\label{module:pyzufall.satz.satz_absurde_farbfunktion}\pysiglinewithargsret{\code{pyzufall.satz.}\bfcode{satz\_absurde\_farbfunktion}}{}{}
Generiert einen Satz nach folgendem Muster: Gelb ist brauner als Türkis.

\end{fulllineitems}

\index{satz\_adjektiv\_am\_ort() (in Modul pyzufall.satz)}

\begin{fulllineitems}
\phantomsection\label{module:pyzufall.satz.satz_adjektiv_am_ort}\pysiglinewithargsret{\code{pyzufall.satz.}\bfcode{satz\_adjektiv\_am\_ort}}{}{}
Generiert einen Satz nach dem Muster: \textless{}Ort\textgreater{} \textless{}Verb\textgreater{} \textless{}Person\textgreater{} \textless{}Adjektiv\textgreater{}.

Beispiel: Auf dem Spielplatz ist die Freundin hilfsbereit.

\end{fulllineitems}

\index{satz\_adjektiv\_sprichwort() (in Modul pyzufall.satz)}

\begin{fulllineitems}
\phantomsection\label{module:pyzufall.satz.satz_adjektiv_sprichwort}\pysiglinewithargsret{\code{pyzufall.satz.}\bfcode{satz\_adjektiv\_sprichwort}}{}{}
Generiert einen Satz nach dem Muster: Je untrainierter desto lächerlicher.

\end{fulllineitems}

\index{satz\_arbeit() (in Modul pyzufall.satz)}

\begin{fulllineitems}
\phantomsection\label{module:pyzufall.satz.satz_arbeit}\pysiglinewithargsret{\code{pyzufall.satz.}\bfcode{satz\_arbeit}}{}{}
Generiert einen Satz über eine berufstätige Person.

Beispiel: Achmed, der Grafiker aus Waldheim, spielt den Nasenbär.

\end{fulllineitems}

\index{satz\_band() (in Modul pyzufall.satz)}

\begin{fulllineitems}
\phantomsection\label{module:pyzufall.satz.satz_band}\pysiglinewithargsret{\code{pyzufall.satz.}\bfcode{satz\_band}}{}{}
Generiert einen Satz zum Thema Band.

\end{fulllineitems}

\index{satz\_band\_besetzung() (in Modul pyzufall.satz)}

\begin{fulllineitems}
\phantomsection\label{module:pyzufall.satz.satz_band_besetzung}\pysiglinewithargsret{\code{pyzufall.satz.}\bfcode{satz\_band\_besetzung}}{}{}
Generiert einen Satz mit den Mitgliedern einer Band.

Beispiel: Die Black Metal Band ``Die Oralen Nudeln'' besteht aus Marlene, Gert, Stefanie, Timm, Andrej, Friederike und Dorothea.

\end{fulllineitems}

\index{satz\_band\_gegruendet() (in Modul pyzufall.satz)}

\begin{fulllineitems}
\phantomsection\label{module:pyzufall.satz.satz_band_gegruendet}\pysiglinewithargsret{\code{pyzufall.satz.}\bfcode{satz\_band\_gegruendet}}{}{}
Generiert einen Satz, der den Zeitpunkt einer Bandgründung zum Thema hat.

Beispiel: Die Electroband ``Kartoffel auf dem Klo'' wurde am 26.10.2009 in Selb gegründet.

\end{fulllineitems}

\index{satz\_band\_mitglied() (in Modul pyzufall.satz)}

\begin{fulllineitems}
\phantomsection\label{module:pyzufall.satz.satz_band_mitglied}\pysiglinewithargsret{\code{pyzufall.satz.}\bfcode{satz\_band\_mitglied}}{}{}
Generiert einen Satz, in dem ein Bandmitglied vorgestellt wird.

Beispiel: Annelise ist Gitarristin von der Gothicband ``Kräuter in der Innenstadt''.

\end{fulllineitems}

\index{satz\_baum() (in Modul pyzufall.satz)}

\begin{fulllineitems}
\phantomsection\label{module:pyzufall.satz.satz_baum}\pysiglinewithargsret{\code{pyzufall.satz.}\bfcode{satz\_baum}}{}{}
Generiert einen Satz mit dem Thema Baum.

Beispiel: Die gnadenlose Kerstin tritt gegen den Apfelbaum.

\end{fulllineitems}

\index{satz\_essen() (in Modul pyzufall.satz)}

\begin{fulllineitems}
\phantomsection\label{module:pyzufall.satz.satz_essen}\pysiglinewithargsret{\code{pyzufall.satz.}\bfcode{satz\_essen}}{}{}
Generiert einen Satz mit Essen und/oder Trinken.

Beispiel: Die Wärterin isst Orangen mit Mayonnaise und trinkt dazu Milch.

\end{fulllineitems}

\index{satz\_farbe() (in Modul pyzufall.satz)}

\begin{fulllineitems}
\phantomsection\label{module:pyzufall.satz.satz_farbe}\pysiglinewithargsret{\code{pyzufall.satz.}\bfcode{satz\_farbe}}{}{}
Generiert einen Satz nach dem Muster: Braun ist eine unsittliche Farbe.

\end{fulllineitems}

\index{satz\_folgehandlung() (in Modul pyzufall.satz)}

\begin{fulllineitems}
\phantomsection\label{module:pyzufall.satz.satz_folgehandlung}\pysiglinewithargsret{\code{pyzufall.satz.}\bfcode{satz\_folgehandlung}}{}{}
Generiert einen Satz, der eine Folgehandlung beschreibt.

Beispiel: Ohne dass Irmgard überlebt, bricht sie aggressiv ein.

\end{fulllineitems}

\index{satz\_frage() (in Modul pyzufall.satz)}

\begin{fulllineitems}
\phantomsection\label{module:pyzufall.satz.satz_frage}\pysiglinewithargsret{\code{pyzufall.satz.}\bfcode{satz\_frage}}{}{}
Generiert eine zufällige Frage.

\end{fulllineitems}

\index{satz\_frage\_1() (in Modul pyzufall.satz)}

\begin{fulllineitems}
\phantomsection\label{module:pyzufall.satz.satz_frage_1}\pysiglinewithargsret{\code{pyzufall.satz.}\bfcode{satz\_frage\_1}}{}{}
Generiert eine Frage nach dem Grund, aus dem eine Person eine Tätigkeit ausführt

Beispiel: Wieso fällt dein Partner in Gedanken hin?

\end{fulllineitems}

\index{satz\_frage\_2() (in Modul pyzufall.satz)}

\begin{fulllineitems}
\phantomsection\label{module:pyzufall.satz.satz_frage_2}\pysiglinewithargsret{\code{pyzufall.satz.}\bfcode{satz\_frage\_2}}{}{}
Generiert eine Frage nach der Person, die eine Tätigkeit ausführt.

Beispiel: Wer telefoniert bewusstlos in der Abtei?

\end{fulllineitems}

\index{satz\_frage\_3() (in Modul pyzufall.satz)}

\begin{fulllineitems}
\phantomsection\label{module:pyzufall.satz.satz_frage_3}\pysiglinewithargsret{\code{pyzufall.satz.}\bfcode{satz\_frage\_3}}{}{}
Generiert eine Frage nach dem Ort, an dem eine Person eine Tätigkeit ausführt.

Beispiel: Wo singt ein Siebdrucker?

\end{fulllineitems}

\index{satz\_frage\_4() (in Modul pyzufall.satz)}

\begin{fulllineitems}
\phantomsection\label{module:pyzufall.satz.satz_frage_4}\pysiglinewithargsret{\code{pyzufall.satz.}\bfcode{satz\_frage\_4}}{}{}
Generiert eine Frage nach der Art, wie eine Person eine Tätigkeit ausführt.

Beispiel: Wie wird sie beim ersten Date angefasst?

\end{fulllineitems}

\index{satz\_frage\_5() (in Modul pyzufall.satz)}

\begin{fulllineitems}
\phantomsection\label{module:pyzufall.satz.satz_frage_5}\pysiglinewithargsret{\code{pyzufall.satz.}\bfcode{satz\_frage\_5}}{}{}
Generiert eine Frage nach dem Zeitpunkt, an dem eine Person eine Tätigkeit ausführt.

Beispiel: Wann säuft eine Hure?

\end{fulllineitems}

\index{satz\_freunde\_lieben() (in Modul pyzufall.satz)}

\begin{fulllineitems}
\phantomsection\label{module:pyzufall.satz.satz_freunde_lieben}\pysiglinewithargsret{\code{pyzufall.satz.}\bfcode{satz\_freunde\_lieben}}{}{}
Generiert einen Satz über eine Person mit Eigenschaften.

Beispiel: In der Garage ist das Mannsweib lesbisch.

\end{fulllineitems}

\index{satz\_kloster() (in Modul pyzufall.satz)}

\begin{fulllineitems}
\phantomsection\label{module:pyzufall.satz.satz_kloster}\pysiglinewithargsret{\code{pyzufall.satz.}\bfcode{satz\_kloster}}{}{}
Generiert einen Satz über eine Person in einem Kloster.

Beispiel: Der Bruder Florian ist der debilste Mönch in der Abtei.

\begin{notice}{note}{Zu tun}

''...''ste nicht immer richtig:
\begin{itemize}
\item {} 
Bruder Dennis war der monströsste Mönch im Kloster.

\item {} 
Schwester Lara ist die diskretste Nonne in der Abtei.

\item {} 
Bruder Marcel war der gerechtste Mönch im Orden.

\item {} 
Bruder Nicolaus ist der falschste Mönch im Kloster.

\end{itemize}
\end{notice}

\end{fulllineitems}

\index{satz\_koerperteil() (in Modul pyzufall.satz)}

\begin{fulllineitems}
\phantomsection\label{module:pyzufall.satz.satz_koerperteil}\pysiglinewithargsret{\code{pyzufall.satz.}\bfcode{satz\_koerperteil}}{}{}
Generiert einen Satz zum Thema Körperteile.

Beispiel: Die ekelhafte Oma massiert ihren Fuß.

\end{fulllineitems}

\index{satz\_nulltransitiv() (in Modul pyzufall.satz)}

\begin{fulllineitems}
\phantomsection\label{module:pyzufall.satz.satz_nulltransitiv}\pysiglinewithargsret{\code{pyzufall.satz.}\bfcode{satz\_nulltransitiv}}{}{}
Generiert einen Satz mit einem nulltransitiven Verb.

Beispiel: Im Park schneit es.

\end{fulllineitems}

\index{satz\_standard() (in Modul pyzufall.satz)}

\begin{fulllineitems}
\phantomsection\label{module:pyzufall.satz.satz_standard}\pysiglinewithargsret{\code{pyzufall.satz.}\bfcode{satz\_standard}}{}{}
Generiert einen zufälligen Standard-Satz.

\end{fulllineitems}

\index{satz\_standard\_1() (in Modul pyzufall.satz)}

\begin{fulllineitems}
\phantomsection\label{module:pyzufall.satz.satz_standard_1}\pysiglinewithargsret{\code{pyzufall.satz.}\bfcode{satz\_standard\_1}}{}{}
Generiert einen einfachen Satz nach dem Muster: \textless{}Person\textgreater{} \textless{}Verb\textgreater{} \textless{}Adjektiv\textgreater{} \textless{}Ort\textgreater{}.

Beispiel: Die Geschmacklose bepisst sich cool in der Kirche.

\end{fulllineitems}

\index{satz\_standard\_2() (in Modul pyzufall.satz)}

\begin{fulllineitems}
\phantomsection\label{module:pyzufall.satz.satz_standard_2}\pysiglinewithargsret{\code{pyzufall.satz.}\bfcode{satz\_standard\_2}}{}{}
Generiert einen einfachen Satz nach dem Muster: \textless{}Ort\textgreater{} \textless{}Verb\textgreater{} \textless{}Person\textgreater{} \textless{}Adjektiv\textgreater{}.

Beispiel: Beim ersten Date flieht er.

\end{fulllineitems}

\index{satz\_standard\_3() (in Modul pyzufall.satz)}

\begin{fulllineitems}
\phantomsection\label{module:pyzufall.satz.satz_standard_3}\pysiglinewithargsret{\code{pyzufall.satz.}\bfcode{satz\_standard\_3}}{}{}
Generiert einen einfachen Satz nach dem Muster: \textless{}Adjektiv\textgreater{} \textless{}Verb\textgreater{} \textless{}Person\textgreater{} \textless{}Ort\textgreater{}.

Beispiel: Gehirntot weint die Schädlingsbekämpferin in der Psychiatrie.

\end{fulllineitems}

\index{satz\_standard\_4() (in Modul pyzufall.satz)}

\begin{fulllineitems}
\phantomsection\label{module:pyzufall.satz.satz_standard_4}\pysiglinewithargsret{\code{pyzufall.satz.}\bfcode{satz\_standard\_4}}{}{}
Generiert einen einfachen Satz nach dem Muster: \textless{}Person\textgreater{} \textless{}Verb\textgreater{} \textless{}Person/Objekt\textgreater{} \textless{}Adjektiv\textgreater{} \textless{}Ort\textgreater{}.

Beispiel: Der Ruhige raubt ein Schaf aus.

\end{fulllineitems}

\index{satz\_thema() (in Modul pyzufall.satz)}

\begin{fulllineitems}
\phantomsection\label{module:pyzufall.satz.satz_thema}\pysiglinewithargsret{\code{pyzufall.satz.}\bfcode{satz\_thema}}{}{}
Generiert einen Satz zu einem zufälligen Thema.

\end{fulllineitems}



\renewcommand{\indexname}{Python-Modulindex}
\begin{theindex}
\def\bigletter#1{{\Large\sffamily#1}\nopagebreak\vspace{1mm}}
\bigletter{p}
\item {\texttt{pyzufall.generator}}, \pageref{module:module-pyzufall.generator}
\item {\texttt{pyzufall.helfer}}, \pageref{module:module-pyzufall.helfer}
\item {\texttt{pyzufall.person}}, \pageref{module:module-pyzufall.person}
\item {\texttt{pyzufall.satz}}, \pageref{module:module-pyzufall.satz}
\end{theindex}

\renewcommand{\indexname}{Stichwortverzeichnis}
\printindex
\end{document}
